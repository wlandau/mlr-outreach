\begin{vframe}{About}
  \begin{itemize}
    \item Project home page\\
    \oneliner{\url{https://github.com/mlr-org/mlr}}
      \begin{itemize}
        \item \textbf{Tutorial} for online viewing / download, including many examples
        \item R documentation rendered in HTML
        \item If you are interested you can ask questions in the github issue tracker
        % \item Wiki page for this tutorial (slides, hands on solutions, \ldots)
      \end{itemize}
    \item 8-10 main developers, quite a few contributors, 4 GSOC projects in 2015/16 and 
    one coming in 2017
    \item About 20K lines of code, 8K lines of unit tests
    % \item If you do not have \pkg{mlr} installed yet, please do so (see wiki page)
      % \item Same for \pkg{OpenML} (not on CRAN, you'll need \pkg{devools}):
% <<openml-install,eval=FALSE>>=
% install.packages("devtools")
% devtools::install_github("openml/r")
% @
  \end{itemize}
\end{vframe}

% \begin{vframe}{Overview}
  % \tableofcontents
% \end{vframe}

% \begin{vframe}
%   \begin{blocki}{What is (supervised) machine learning?}
%   \item Learning structure in data:\\
%     Classification, regression, survival analysis, clustering, $\ldots$
%   \item The art of predicting stuff
%   \item Model optimization
%   \item Understanding of grey-box models
%   \end{blocki}
% 
%   \begin{blocki}{Disclaimer}
%   \item The list is subjective and naively tailored to this talk
%   \item ML is based on math and statistics, we will (mainly) talk about structure, software, and practical issues here
%   \end{blocki}
% \end{vframe}



% \begin{vframe}{Supervised Classification tasks}
% <<classification-task-plot,echo=FALSE,fig.height=4>>=
% set.seed(1)
% df = data.frame(x = c(rnorm(10, mean = 3), rnorm(10, mean = 5)), y = runif(10), class = rep(c("a", "b"), each = 10))
% ggplot(df, aes(x = x, y = y, shape = class, color = class)) + geom_point(size = 3) + geom_vline(xintercept = 4, linetype = "longdash")
% @
% \structure{Goal}: Predict a class (or membership probabilities)
% \end{vframe}


% \begin{vframe}{Supervised Regression tasks}
% <<regression-task-plot,echo=FALSE,fig.height=4>>=
% set.seed(1)
% f = function(x) 0.5 * x^2 + x + sin(x)
% x = runif(40, min = -3, max = 3)
% y = f(x) + rnorm(40)
% df = data.frame(x = x, y = y)
% ggplot(df, aes(x, y)) + geom_point(size = 3) + stat_function(fun = f, color = "#FF9999", size = 2)
% @
% \structure{Goal}: Predict a continuous output
% \end{vframe}


% \begin{vframe}{Supervised Survival tasks}
% <<survial-task-plot,echo=FALSE,fig.height=4>>=
% set.seed(1)
% data("rats", package = "survival")
% sf = survfit(Surv(time, status) ~ rx, data = rats)
% survMisc:::autoplot.survfit(sf, title = "", xLab = "Time", yLab = "$\\hat{S}(t)$\n", survLineSize = 1.5)$plot
% @
% \structure{Goal}: Predict a survival function $\hat{S}(t)$, i.e.\ the probability to survive to time point~$t$
% \end{vframe}


% \begin{vframe}{Unsupervised Cluster tasks}
% <<cluster-task-plot,echo=FALSE,fig.height=4>>=
% df = iris
% m = as.matrix(cbind(df$Petal.Length, df$Petal.Width),ncol=2)
% cl = (kmeans(m,3))
% df$cluster = factor(cl$cluster)
% centers = as.data.frame(cl$centers)
% ggplot(data=df, aes(x=Petal.Length, y=Petal.Width, color=cluster )) +
%  geom_point() +
%  geom_point(data=centers, aes(x=V1,y=V2, color='Center')) +
%  geom_point(data=centers, aes(x=V1,y=V2, color='Center'), size=52, alpha=.3) +
%  theme(legend.position="none")
% @
% \structure{Goal}: Group data into similar clusters (or estimate fuzzy membership probabilities)
% \end{vframe}


% \section{Why mlr?}
\begin{vframe}{Motivation}
  \begin{blocki}{The good news}
  \item CRAN serves hundreds of packages for machine learning
    % (cf.\ CRAN task view machine learning)
  \item Often compliant to the unwritten interface definition:
\begin{knitrout}
\definecolor{shadecolor}{rgb}{0.969, 0.969, 0.969}\color{fgcolor}\begin{kframe}
\begin{alltt}
\hlstd{model} \hlkwb{=} \hlkwd{fit}\hlstd{(target} \hlopt{~} \hlstd{.,} \hlkwc{data} \hlstd{= train.data, ...)}
\hlstd{predictions} \hlkwb{=} \hlkwd{predict}\hlstd{(model,} \hlkwc{newdata} \hlstd{= test.data, ...)}
\end{alltt}
\end{kframe}
\end{knitrout}
  \end{blocki}
% \end{vframe}

% \begin{vframe}{Motivation}
  \begin{blocki}{The bad news}
    \item Some packages API is \enquote{just different}
    \item Functionality is always package or model-dependent, even though the procedure might be general
    \item No meta-information available or buried in docs 
      % (sometimes not documented at all)
    % \item Many packages require the user to \enquote{guess} good hyperparameters
    % \item Result: engthy, tedious and error-prone code
  \end{blocki}
  \oneliner{Our goal: A domain-specific language for many machine learning concepts!}
\end{vframe}


\begin{vframe}{Motivation: \pkg{mlr}}
  \begin{itemize}
    \item Unified interface for the basic building blocks: tasks, learners, resampling, hyperparameters, \ldots
    \item Reflections: nearly all objects are queryable (i.e.\ you can ask them for their properties and program on them)
    \item The OO-structure allows many generic algorithms:
      \begin{itemize}
        \item Bagging
        \item Stacking
        \item Feature Selection
        \item \ldots
      \end{itemize}
    \item Easily extensible via S3
      \begin{itemize}
        \item Extension is not covered here, but explained in detail in the online tutorial
        \item You do not need to understand S3 to use \pkg{mlr}
        \item Wondering why we don't use S4? We care about code bloat and speed.
      \end{itemize}
  \end{itemize}
\end{vframe}

\begin{vframe}{}
\begin{knitrout}
\definecolor{shadecolor}{rgb}{0.969, 0.969, 0.969}\color{fgcolor}\begin{kframe}


{\ttfamily\noindent\bfseries\color{errorcolor}{\#\# Error in plotLearnerPrediction(learner = "{}classif.ksvm"{}, task = iris.task): could not find function "{}plotLearnerPrediction"{}}}\end{kframe}
\end{knitrout}

\end{vframe}

% \begin{vframe}{Some remarks on style}
%   \begin{blocki*}
%   \item Function names are camel-case: doThatThing()
%   \item Arguments and variables are lower-case, with dots: doThatThing(my.arg, another.one)
%   \item We use \enquote{\code{=}} not \enquote{\code{<-}}
%   \item We document in a pretty formal fashion, including type info
%   \item We try to use \enquote{@family} to group functions in the docs
%   \item We try to arg- and user-error-check in the most safe and informative way
%   \end{blocki*}
% \end{vframe}

% \section{Building Blocks}








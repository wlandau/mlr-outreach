\documentclass{article}

\usepackage{cmbright}
\usepackage{hyperref}
\usepackage{parskip}
\setlength{\parindent}{0pt}



\title{Tutorial Proposal: Applied Machine Learning and Efficient Model Selection with mlr}

\author{Bernd Bischl, Michel Lang\\
Department of Statistics, TU Dortmund University}

\date{}

\begin{document}

\maketitle

\paragraph{Brief description of the tutorial}
R does not define a standardized interface for all its machine learning algorithms. Therefore, for
any non-trivial experiments you need to write lengthy, tedious and error-prone wrappers to call the
different algorithms and unify their respective output. The mlr package offers a clean, easy-to-use
and flexible domain specific language for machine learning experiments in R. It supports
classification, regression, clustering and survival analysis and connects to nearly a hundred
predictive modeling techniques. The package allows for different hyperparameter optimization and
configuration techniques, including iterated F-racing and sequential model based optimization.
Variable selection is possible through various filter and wrapper approaches. 

Hence, mlr allows data analysts who are neither experts in machine learning nor seasoned
R programmers to nevertheless specify and complex machine learning experiments in short, succinct
and scalable code. Experienced programmers, on the other hand, get to wield a large, well-designed toolbox, 
which they can customize and extend to quickly construct their own algorithms.

The course will enable the participants to understand and apply the basic mlr operations for data
handling and preprocessing, model building, evaluation and resampling. After these basics are
covered we will especially focus on the important aspects of benchmarking, model selection and
hyperparameter tuning. As all of these usually require a large amount of computational resources in
realistic applications, we will show how to easily parallelize them in common parallel environments.
The course will end with a short demonstration on how to access the new OpenML server for open
machine learning (\url{http://www.openml.org}) which provides a large repository of benchmark data sets 
and enables reproducible experiments and meta analysis.

Project page: \url{https://www.github.com/berndbischl/mlr/}\\
Online tutorial: \url{https://berndbischl.github.io/mlr/tutorial/html/}

\pagebreak

\paragraph{Detailed Outline}
\begin{itemize}
\item Very brief intro to applied machine learning
\item Data handling and machine learning tasks 
\item Classification, regression, clustering and survival modeling with mlr
\item Performance evaluation and resampling
\item Visual model analysis
\item Parallelization and high-performance computing
\item Model selection and hyper-parameter tuning
\item Interfacing the OpenML server with mlr
\end{itemize}

\paragraph{Background knowledge required} Basic knowledge of R and machine learning

\paragraph{Potential attendees} Anybody from academia or industry with an interest in modern machine learning

\end{document}

